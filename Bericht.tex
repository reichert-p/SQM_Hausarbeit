\documentclass[12pt]{article}
\setcounter{secnumdepth}{-1}
\setlength{\parindent}{0em}


\begin{document}

\title{Fallbeispiel}
\date{01.05.2022}
\author{Philipp Reichert}

\maketitle

\section{Vorstellung des Fallbeispiels}



\section{Betrachtung des SQM im Fallbeispiel}

1) Gibt es einen definierten Prozess?
2) Gibt es einen definierten Qualitätsanspruch?
3) Gibt es ein definiertes QM-System?

Beantwortet diese Fragen in der Betrachtung, vertieft dann begründet einen dieser Aspekte

\section{Analyse und Bewertung des SQM im Fallbeispiel}

1) Gibt es eine/mehrere definierte Methoden?
2) Wenn ja, welche und mit welcher Verwendung?
3) Wie sind die Methoden in einen Prozess integriert?

\section{Erarbeitung von Optimierungsmaßnahmen}

1. Was ist das Problem?
Die Zusammenarbeit der Abteilungen innerhalb des Unternehmens wird negativ
wahrgenommen. Ursache dafür ist die fehlende Transparenz über das QM und das dadurch
entstehende Misstrauen.
2. Warum wollen Sie es lösen?
Um die Zusammenarbeit zu fördern.
3. Wie wollen Sie es lösen?
QM Maßnahmen analysieren und Optimierungsmaßnahmen einführen

\end{document}


%Ziel der vorliegenden Arbeit ist es,
%• QM anhand etablierter Grundsätze und Methoden zu vergleichen,
%%• marktspezifisch zu bewerten
%• und mit zielgerichteten Maßnahmen zu optimieren
%• um die Wahrnehmung der geleisteten Qualität der IT-Dienstleistungen zu steigern
%• und so eine enge Zusammenarbeit innerhalb der Unternehmensgruppe zu fördern.