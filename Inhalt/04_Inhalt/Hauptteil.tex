\chapter{Hauptteil}

\section{Priorität des Projekts}


Die standardisierte Datenkommunikation spielt eine wichtige Rolle bei der Zusammenarbeit mehrerer getrennt verwalteter Institutionen und gewährleistet einen reibungslosen Ablauf von Prozessen. Da sich Prozesse im Laufe der Zeit weiterentwickeln, müssen auch die Kommunikationsstandards gelegentlich angepasst werden.

Die Unterstützung der neuesten Standards ist eine grundlegende funktionale Anforderung für ERP-Software, da sie die aktuelle und zukünftige reibungslose Kommunikation mit anderen Institutionen gewährleistet.
Die Umsetzung dieser Anforderung ist von großer Bedeutung für das Gesamtprodukt.

Um zu zeigen, dass die Richtlinien unterstützt werden, soll das entsprechende Zertifikat von Swissdec erworben werden. Dadurch wird das Qualitätsmanagement vereinfacht, da einheitliche Vorgaben eingehalten werden, es sorgt für Transparenz bei Kunden und bietet eine objektive Überprüfung des Erfolges des Projekts

Es ist jedoch wichtig zu beachten, dass dieses Qualitätsmerkmal erst bei Abschluss des Projekts erworben werden kann und nichts über den tatsächlichen Verlauf des Projekts aussagt. Daher sind weitere Maßnahmen im Softwarequalitätsmanagement erforderlich.

\subsection{Ressourcen}

Das Projektteam besteht aus zwei erfahrenen Entwicklern, die bereits seit über 10 Jahren an der Software arbeiten, einem Junior-Entwickler und einem dualen Studenten.  Ihre Aufgaben umfassen sowohl die Implementierung des neuen Standards als auch die Wartung der bestehenden Lohndatenabrechnung.

Das Entwicklungsteam wird von zwei Personen unterstützt, die sich mit Fragen zur Lohndatenabrechnung, den rechtlichen Bestimmungen in der Schweiz und Rückfragen an Swissdec befassen. Ihr Verantwortungsbereich umfasst auch manuelle Ende-zu-Ende-Tests für durch Rückfragen abgeklärte Randfälle.

Der Zeitliche Rahmen für die Umsetzung des Projektes liegt bei zwei Jahren.

\section{Aktuelles Vorgehen, dessen Hintergründe und Auswirkungen}

\subsection{Systemarchitektur}

Die Codebasis erstreckt sich über eine Vielzahl von Systemen, die unabhängige Entwicklungsstände und Testdaten aufweisen. 
Es gibt vier Entwicklungssysteme, auf denen der neue Lohnmeldestandard implementiert werden soll. Diese unterscheiden sich in den Implementierungsdetails, da sie jeweils verschiedene ABAP-Versionen verwenden.

Der Transport von Versionsänderungen zwischen den Systemen ist teilweise automatisch möglich, kann jedoch bis zu 40 Minuten dauern und erfordert möglicherweise manuelle Codeanpassungen.

Für jedes Entwicklungssystem gibt es mindestens ein Testsystem. Es können auch mehrere Testsysteme mit unterschiedlichen Systemeinstellungen und Testdaten vorhanden sein. Die Anzahl der Testsysteme hängt davon ab, wie viele Kunden die entsprechende ABAP-Version mit verschiedenen Systemeinstellungen nutzen.

Diese Vielfalt an Systemen ergibt sich aus der SAP-Politik, alte Versionen weit über ihr technologisches Haltbarkeitsdatum hinaus zu unterstützen.

Diese verteilte Entwicklung und Testung beeinträchtigt die Effizienz der Entwickler, da sie ähnliche Algorithmen mitunter mehrfach entwickeln müssen und die Selben Tests mehrfach an verschiedenen Stellen ausgeführt werden müssen.
Zudem nimmt der Transport in die verschiedenen Systeme Zeit in Anspruch. Darüber hinaus kann es lange dauern, Feedback zu Fehlern zu erhalten, wenn die zugehörigen Tests in anderen Systemen liegen.

\subsection{Auslieferung}

Es wird alle drei Monate ein Software-Update ausgeliefert. Dazwischen können auch schnellere Fehlerbehebungen ausgeliefert werden. Die Entscheidung, ob Kunden diese Updates installieren möchten oder lieber auf das nächste große Update warten möchten, liegt bei ihnen selbst.

Dies entsteht aus der Politik von SAP, die Auslieferung von Software nicht zentral zu verwalten, sondern dies den einzelnen Kunden auf ihren On-Premise-Systemen zu überlassen.

Dies führt dazu, dass Kunden auf verschiedenen Versionen operieren, was im Fall von kundenseitig entdeckten Fehlern aufwendig nachgestellt werden muss.
 Dies kostet viel Zeit und führt oft zu zusätzlichen Problemen und Schwierigkeiten bei der Replikation der Zustände.
 
\subsection{Software Tests}

Es existieren spezifische Testfälle, die von Swissdec definiert wurden und für die Zertifizierung korrekt durchgeführt werden müssen. Sobald die entsprechenden Features implementiert wurden, werden diese Testfälle in die jeweiligen Testsysteme integriert und umgesetzt. Ein Test gilt als bestanden, wenn die Lohnabrechnung der Testfälle korrekt über den vorgesehenen manuellen Prozess erstellt werden kann.

Dieses Vorgehen ergibt sich aus dem allgemeinen Charakter der Tests für die Zertifizierung des ELM-Standards. Es wird lediglich die korrekte Formatierung der Daten geprüft, nicht jedoch deren Erstellung.

Die manuelle führt zu einer fehlenden Übersicht darüber, welche Tests gerade auf welchem Stand durchgeführt werden.
Das Fehlen detaillierter Tests bedeutet, dass es nicht möglich ist, im Detail zu überprüfen, ob die Software wie geplant funktioniert.
Darüber hinaus werden Nebeneffekte oft erst spät bemerkt, da keine automatisierten Tests vorhanden sind. Dies ist problematisch, da die Kosten von Fehlern umso höher sind, je später sie entdeckt werden.

\subsection{Projektmanagement}

Die Zuweisung der Aufgaben erfolgt über ein dynamisches Dokument, indem ausstehende Aufgaben aufgelistet werden.
Die Zuteilung der Aufgaben sowie Rückmeldungen zu Fortschritten findet in täglichen Meetings statt. Es gibt jedoch kein klar definiertes Verfahren dafür.

Dies liegt an der hohen Erfahrungsdifferenz und der damit automatisch verbundenen Hierarchie in der Abteilung.

Dies stellt jedoch ein Problem dar, da aufgrund der langen Veröffentlichungsintervalle schwierig Rückschlüsse zum aktuellen Entwicklungsfortschritt gezogen werden können.
Darüber hinaus ist die Einschätzung der verbleibenden Zeit im Projekt nicht sehr zuverlässig, da mit ähnlicher Geschwindigkeit neue Aufgaben gefunden werden, wie alte Aufgaben abgearbeitet werden.

\subsection{Dokumentation und Versionskontrolle}

Wenn eine neue Aufgabe begonnen wird, wird dafür eine Feature-Note erstellt, auf der alle vorgenommenen Codeänderungen gespeichert werden.
Die geplante Funktionalität wird ebenfalls auf der Note beschrieben. Zusätzlich wird am Anfang der geänderten Dateien eine Referenz auf die Feature-Note und den Entwickler hinterlegt.

Die Einführung von Feature-Notes erfolgte, um Release-Dokumentationen aus den Feature-Notes zu generieren. Dieses Konzept besteht seit den frühen Anfängen des ERP-Systems und wurde bereits verwendet, bevor moderne Projektmanagement-Tools in anderen Teilen des Unternehmens eingesetzt wurden.
 
Die Notizen am Anfang der Dateien haben ebenfalls ihren Ursprung in den Anfängen des Produkts. Sie wurden eingeführt, damit Autor und Zweck des Codes auch ohne moderne Versionskontrolle erkannt wurden.
Diese Anmerkungen existieren seit langem und wurden teilweise sogar auf Deutsch oder Italienisch verfasst, obwohl die Implementierungssprache seit Jahrzehnten einheitlich englisch ist.

Dies führt zu Problemen, da bei häufig bearbeiteten Dateien ein unübersichtlicher Overhead am Anfang der Datei entsteht und die Verantwortlichen, die durch Mitarbeiternummern repräsentiert werden, teilweise seit Jahrzehnten nicht mehr im Unternehmen tätig sind.

\subsection{Zusammenfassung der Probleme}

Der Kern der Probleme liegt in der Änderung der Zusammenstellung des Teams.
Die erfahrenen Entwicklern, haben lange Zeit ohne besondere Projektmanagementtätigkeiten als Zweierteam gearbeitet und sehen keinen Anlass bei dem erweiterten Team und größeren Projekt etwas grundlegend an ihren Arbeitsabläufen zu ändern.

 Zudem besteht keine klare Verantwortlichkeit, wessen Aufgabe es wäre Änderungen in die Hand zu nehmen. %TODO ?

\section{Grundlagentheorie für Optimierung}

Um Optimierungen für das Softwarequalitätsmanagement des Projekts vorzuschlagen, ist es wichtig, sich vorher mit den in der Vorlesung erlernten Grundlagen auseinander zu setzen.

\subsection{PDCA}

PDCA steht für den Plan-Do-Check-Act-Zyklus, der als ein strukturierter Ansatz zur kontinuierlichen Verbesserung von Prozessen, Produkten oder Dienstleistungen verwendet wird. Der PDCA Kreis besteht dabei aus vier Phasen

In der Planungsphase werden Ziele festgelegt und ein Aktionsplan erstellt. Es werden die zu erreichenden Ergebnisse, Ressourcenanforderungen und Maßnahmen definiert. 
Bei der Umsetzung (Do) Phase wird der im Planungsprozess entwickelte Aktionsplan umgesetzt. Es werden Maßnahmen ergriffen, um die gewünschten Verbesserungen in der Praxis umzusetzen. 
 In der Überprüfungsphase werden die durchgeführten Maßnahmen überprüft und die erzielten Ergebnisse analysiert. Es werden Daten gesammelt und Leistungskennzahlen verwendet, um den Fortschritt zu bewerten und festzustellen, ob die angestrebten Ziele erreicht wurden.
Basierend auf den Ergebnissen der Überprüfung wird gehandelt (Act), also Anpassungen vorgenommen und Verbesserungsmaßnahmen identifiziert. Wenn die gewünschten Ziele erreicht wurden, werden die erfolgreich umgesetzten Maßnahmen standardisiert und in den laufenden Betrieb überführt. Falls notwendig, werden neue Pläne entwickelt und der PDCA-Zyklus beginnt erneut.

Der PDCA-Zyklus stellt sicher, dass kontinuierliche Verbesserungen in einem strukturierten und wiederholbaren Prozess stattfinden. Durch die wiederholte Anwendung des Zyklus können Projekte kontinuierlich lernen, ihre Prozesse optimieren und eine  Verpflichtung zur Verbesserung etablieren.

\subsection{Wasserfall Modell und Kanban Board}

Das Wasserfallmodell ist ein sequenzieller und linearer Ansatz zur Softwareentwicklung. Es basiert auf der Idee, dass jeder Schritt im Entwicklungsprozess vollständig abgeschlossen sein muss, bevor der nächste Schritt beginnt. Der Name "Wasserfall" bezieht sich auf die Vorstellung, dass der Fortschritt von oben nach unten fließt, ähnlich einem Wasserfall.

Im Wasserfallmodell durchläuft ein Projekt typischerweise die folgenden Phasen in fester Reihenfolge: Anforderungsdefinition, Systementwurf, Implementierung, Überprüfung und Wartung. Jede Phase wird abgeschlossen, bevor die nächste beginnt, und es gibt wenig bis gar keine Möglichkeit für Rückwärtsbewegungen oder iterative Schleifen. Das bedeutet, dass jede Änderung oder Neuanforderung, die nach Abschluss einer Phase auftritt, erst in späteren Phasen berücksichtigt werden kann.

Das Wasserfallmodell eignet sich gut für Projekte mit klaren und stabilen Anforderungen, bei denen die Ergebnisse im Voraus genau definiert werden können, so etwa wenn ein Zertifikat erworben werden soll.
Es bietet eine klare Struktur und ermöglicht eine detaillierte Planung. Allerdings kann es bei komplexen Projekten oder solchen mit sich ändernden Anforderungen unflexibel und risikoreich sein. Fehler oder Mängel, die in einer frühen Phase übersehen werden, können sich im späteren Verlauf des Projekts schwerwiegend auswirken.

Ein Kanban Board ist ein visuelles Tool zur Verwaltung von Arbeitsabläufen und Aufgaben ist. Ein Kanban Board besteht aus Spalten, die den verschiedenen Status einer Aufgabe oder eines Arbeitsvorgangs repräsentieren, und Karten, die die einzelnen Aufgaben darstellen.

Das Kanban Board ermöglicht es einem Team, den Überblick über den Fortschritt der Arbeit zu behalten, Engpässe zu identifizieren und die Prioritäten zu setzen. Jede Aufgabe wird auf einer Karte dargestellt, die je nach Fortschritt durch die Spalten des Boards verschoben wird, von der Aufgabenplanung über die Bearbeitung bis zur Fertigstellung. Teammitglieder können den Status einer Aufgabe schnell erkennen und wissen, welche Aufgaben als nächstes erledigt werden müssen.

Das Kanban Board fördert die Transparenz und Zusammenarbeit im Team. Es ermöglicht eine kontinuierliche Verbesserung, da Engpässe und Flaschenhälse sichtbar werden und Maßnahmen ergriffen werden können, um diese zu beseitigen. Das Board kann auch dazu dienen, den Arbeitsfluss zu optimieren und die Effizienz zu steigern.

Im Vergleich zum Wasserfallmodell bietet das Kanban Board eine flexible und iterative Herangehensweise an die Arbeit. Es passt sich leicht an sich ändernde Anforderungen oder Prioritäten an und ermöglicht eine kontinuierliche Anpassung und Verbesserung des Arbeitsprozesses. Es wird oft in agilen Entwicklungsumgebungen eingesetzt, in denen eine hohe Flexibilität und schnelle Reaktion auf Veränderungen erforderlich sind.

\section{Optimierungen und Marktspezifische Bewertung}

\subsection{Das Bewusstsein für die Notwendigkeit von SQM Maßnahmen stärken}

Um sicherzustellen, dass alle weiteren Schritte erfolgreich sind, ist es wesentlich, ein Bewusstsein dafür zu schaffen, warum eine solide Projektleitung, regelmäßiges Feedback und eine effektive Koordination eine entscheidende Rolle für die Qualität der Software spielen. Besonders wichtig ist es, den Wert des PDCA-Protokolls (Plan-Do-Check-Act) zu verdeutlichen, da es essentiell für die Prozessverbesserung und damit die Softwarequalität ist.

\subsection{Einführung moderner Versionierungs- und Projektmanagement Tools}

Der erste Schritt bei PDCA ist Plan. Entsprechend ist es empfehlenswert ein Vorgehen für den Projektverlauf zu definieren.

Angesichts der Projekteigenschaften bietet es sich an, sich beim Zertifikatserwerb an der Struktur des Wasserfallmodells zu orientieren, obwohl in der Softwareentwicklung normalerweise agile Methoden verwendet werden.
Dies liegt daran, dass die konkreten Anforderungen innerhalb eines festgelegten Zeitraums erfüllt werden müssen und eine umfassende Dokumentation erforderlich ist, die ohnehin erstellt werden muss.
In diesem Kontext bietet das Wasserfallmodell eine geeignetere Struktur für die Umsetzung der Anforderungen.

Ein Vorteil dieser Umsetzung besteht darin, dass klare Anforderungen bereits von Swissdec festgelegt wurden und sich voraussichtlich nicht grundlegend ändern werden. Die Dokumentation, die ohnehin für die Abnahme des Zertifikats am Ende des Projekts erstellt werden muss, kann gut geplant werden. Dadurch können Abweichungen bei den verfügbaren Ressourcen und dem Zeitplan frühzeitig erkannt und durch den PDCA-Zyklus entsprechend gegengesteuert werden.

Ein weiterer Vorteil ist, dass neben der Anforderungsanalyse auch die Überprüfung der Ergebnisse bereits durch die Rahmenbedingungen von Swissdec gegeben sind.

Da das Projekt auf einer bestehenden Codebasis aufbaut, ist es sinnvoll, sich in der Entwurfsphase intensiv mit der vorhandenen Architektur und den erforderlichen Anpassungen auseinanderzusetzen, um Synergien und Probleme zwischen dem alten und neuen Code effektiv nutzen zu können.

Es wird empfohlen, auf neue, modernere Technologien umzusteigen, unter anderem, um die sinnvolle Umsetzung der Planung im Wasserfallmodell zu ermöglichen und die Dokumentation zu erleichtern.


\subsection{Testgetriebene Entwicklung mit automatisierten Tests}

Eine Möglichkeit zur Optimierung besteht darin, den Testbetrieb zu überarbeiten. Ein wichtiger Schritt zur Verbesserung der Codequalität wäre die Implementierung von Unit Tests, die auf kleinstem Niveau eine schnelle Überprüfung der Codefunktionalität ermöglichen.

Des Weiteren wäre es sinnvoll, die testgetriebene Entwicklung auf die bereits vorhandenen Acceptance Tests von Swissdec anzuwenden. Dieser Ansatz würde die bestehenden Qualitätsmetriken nutzen. Da diese Tests festgelegt und für eine erfolgreiche Zertifizierung erforderlich sind, würde dies nicht zu einem höheren Aufwand oder zusätzlicher Testwartung führen.

Da das Projekt im Zusammenhang mit einem umfangreichen Lohndatensystem steht, ist es ratsam, regelmäßige Smoke- und Integrationstests durchzuführen, um zu überprüfen, ob andere Systemkomponenten von der Implementierung beeinflusst werden.

Eine weitere empfehlenswerte Maßnahme zur Sicherstellung der Qualität besteht darin, automatisierte Tests bei neuen Änderungen einzusetzen. Dadurch wird gewährleistet, dass alle Tests regelmäßig durchgeführt werden und Probleme frühzeitig erkannt werden können.

\subsection{Continuous Integration}

Eine effektive Umsetzung dieser Tests könnte im Rahmen einer Continuous Integration realisiert werden. Durch die Implementierung von Continuous Delivery in andere Entwicklungs- und Testsysteme ließe sich ebenfalls Zeit sparen, da Entwickler nicht mehr halbautomatisch Änderungen zwischen den Systemen transportieren müssten.

Es wäre ratsam, die Pipeline um automatisierte Tests in den anderen Systemen zu erweitern, um Entwickler direkt über Codeänderungen zu informieren, die für diese Systeme erforderlich sind \footnote{Aufgrund der verschiedenen ABAP Versionen}.

Jedoch ist es im Zusammenhang mit der Übertragung des Produktivcodes an Kundenproduktionssysteme nicht sinnvoll, Continuous Deployment einzusetzen, da hier liegt die Kontrolle bei den Kunden liegt.

Angesichts der Tatsache, dass die Entwicklung des neuen Lohnmeldestandards auf den gleichen Systemen wie die aktuelle Lohnmeldesoftware stattfindet, muss darüber nachgedacht werden, wie die Auslieferung erfolgen soll.

Der vorgesehene Zeitrahmen für die Entwicklung des neuen Lohnmeldestandards beträgt etwa 3 Jahre. Dies würde bedeuten, dass die Auslieferung in 12 Releases aufgeteilt wird. Da ein unvollständiger Lohnmeldestandard für Kunden nutzlos ist und nur zu Verwirrung bei den Patchnotes führen würde, wird ein gebündelter Launch vorgeschlagen. Dies bedeutet, dass beim tatsächlichen Launch erhebliche Änderungen vorgenommen werden müssen. Da diese Änderungen jedoch kontinuierlich intern in den neuesten Stand integriert werden, kann ein sogenannter "Integration Hell" vermieden werden.

Da die bestehenden Lohnmeldestandards im System weiterhin vorhanden sind, handelt es sich um eine parallele Launchstrategie, sofern Lohnmeldungen als separate austauschbare Programmteile betrachtet werden können.

\chapter{Zusammenfassung}

Insgesamt lässt sich feststellen, dass die Implementierung des neuen Lohnmeldestandards hauptsächlich durch die langjährige Historie und die damit verbundenen veralteten Entwicklungsprozesse erschwert wird. Einige der identifizierten Probleme sind unlösbar, da sie auf vertraglichen Verpflichtungen gegenüber Kunden beruhen.

Dennoch kann eine Verbesserung der Effizienz, Fortschrittsbewertung und der Entwicklungs- und Softwarequalität erreicht werden, indem feste Phasen gemäß dem Wasserfallmodell eingeführt werden und eine verbesserte Testgetriebene Entwicklung mit besserer Testabdeckung und interner Auslieferung in Kombination mit neuen, automatisierbaren Tools verwendet wird.

Die vorgestellten Maßnahmen bieten Entwicklern und Projektmanagern die Möglichkeit, den Prozess eigenständig weiter zu verbessern und eine positive Veränderung herbeizuführen.