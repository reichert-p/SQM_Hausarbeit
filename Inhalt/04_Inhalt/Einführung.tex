\chapter{Einleitung}



Das Ziel dieser Hausarbeit besteht darin, anhand eines praktischen Fallbeispiels das Softwarequalitätsmanagement (SQM) zu analysieren und Optimierungen dafür zu entwerfen.
Das Fallbeispiel ist dabei die Umsetzung eines neuen Lohnstandards und der damit einhergehenden Swissdec Zertifizierung.
Die Ausarbeitung dient dabei dem Nachweis theoretischen und praktischen Wissens und insbesondere das Verständnis zur praktischen Anwendung des erlernten Wissens. 

Elektronische Lohnabrechnungen sind ein wichtiger Bestandteil moderner Enterprise Resource Planning (ERP) Software für den Bereich Human Resources (HR). So bietet auch meine Firma, die SAP, ERP Systeme mit Möglichkeiten zur elektronischen Lohnabrechnung.
Lohnabrechnungen und Lohndaten sind wichtige Personenbezogene Daten, die in verschiedenen Ländern unterschiedlichen Standards und Regulierungen entsprechen. Zudem müssen sie vielen verschiedenen Stellen und Behörden gemeldet werden, wie Ämtern und Versicherungen. 
Damit das einheitlich und rechtlich funktioniert, hat das Gemeinschaftsprojekt Swissdec für die Schweiz einen Standard entwickelt und vergibt Zertifizierungen für deren Einhaltung \cite{Swissdec.06.05.2023}. 
Abbildung \ref{fig:dh-website} zeigt, dass viele wichtige Schweizer Institutionen über Swissdec Standards Kommunizieren. Darum ist es wichtig, dass ERP Systeme die aktuellen Swissdec Standards unterstützen.
Die zum Projektzeipunkt aktulle Zertifizierungsbasis für Lohndatenverarbeitung ist ELM 5.0. Eine Datenspeicherung nach diesem Standard in das bestehende ERP System zu integrieren ist das Ziel des Pojektes dieser Arbeit. 
Wichtige Schritte sind die Verbesserung der Codequalität, die Anpassung der XML Dokumente und die Implementierung aktuellster Verschlüsselungs- und Privacyverfahren.
 
Um diesen Prozess zu verbessern, sollen erlerntes Wissen und Techniken aus der Vorlesung verwendet werden, um die Abläufe zu analysieren, kritisch zu bewerten und Verbesserungsvorschläge zu formulieren.
 
 Dazu wird zunächst die bestehende Software vorgestellt und %TODO beschreiben was das Vorgehen ist

\begin{figure}
	\centering
	\includegraphics[width=1.\textwidth]{Bilder/grafik_lohndaten_de.png} 
	\caption{Die Abbildung zeigt die Stellen, die Swissdec standardisierte Lohndaten entgegennehmen \cite{Swissdec.06.05.2023}}
	\label{fig:dh-website}
\end{figure} 







