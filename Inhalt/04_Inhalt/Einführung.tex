\chapter{Einleitung}



Das Hauptziel dieser Hausarbeit ist es, das Softwarequalitätsmanagement (SQM) anhand eines konkreten Fallbeispiels zu untersuchen und Verbesserungen dafür zu entwickeln.
Das Fallbeispiel bezieht sich auf die Umsetzung eines neuen Lohnstandards und den Erhalt der damit verbundenen Swissdec-Zertifizierung.
Durch diese Ausarbeitung soll der Erhalt theoretischen und praktischen Wissens nachgewiesen werden, insbesondere das Verständnis für die praktische Anwendung des erlernten Wissens. 

Elektronische Lohnabrechnungen sind ein integraler Bestandteil moderner Enterprise Resource Planning (ERP)-Software für den Bereich Human Resources (HR).
Auch meine Firma bietet ERP-Systeme an, die die Möglichkeit zur elektronischen Lohnabrechnung bieten \footnote{In der Schweiz wird üblicherweise von Lohn statt Gehalt gesprochen}.
In Lohnabrechnungen vorkommende Lohndaten sind wichtige personenbezogene Informationen, die je nach Land unterschiedlichen Standards und Vorschriften entsprechen müssen.
Darüber hinaus müssen sie an verschiedene Stellen und Behörden gemeldet werden, wie beispielsweise Ämtern und Versicherungen. 

Um eine einheitliche und rechtlich konforme Abwicklung zu gewährleisten, hat das gemeinschaftliche Projekt Swissdec einen Standard für die Schweiz entwickelt und vergibt Zertifizierungen zur Einhaltung dieses Standards  \cite{Swissdec.06.05.2023}. 
Abbildung \ref{fig:dh-website} zeigt, dass viele bedeutende schweizerische Institutionen Swissdec Standards akzeptieren. Daher ist es wichtig, dass ERP-Systeme die aktuellen Swissdec Standards unterstützen.

\begin{figure}
	\centering
	\includegraphics[width=200px]{Bilder/grafik_lohndaten_de.png} 
	\caption{Die Abbildung zeigt die Stellen, die Swissdec standardisierte Lohndaten entgegennehmen \cite{Swissdec.06.05.2023}}
	\label{fig:dh-website}
\end{figure} 

Bisher werden die Standards bis ELM 4.0 unterstützt.
Das Ziel dieses Projekts besteht darin, die während des Projektzeitraums aktuellste Zertifizierung für die Verarbeitung von Lohndaten, ELM 5.0, in das bestehende ERP-System zu integrieren.
Wesentliche Schritte dafür sind die Verbesserung der Qualität des existierenden Codes, die Anpassung der generierten XML-Struktur sowie die Implementierung der für die Zertifizierung benötigten Verschlüsselungs- und Datenschutzverfahren.
 










